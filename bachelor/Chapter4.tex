% !TEX root = ../Thesis.tex
\chapter{Conclusion}
The goal of this thesis was to give an in depth and intuitive explanation of PINCH. Additionally I have expanded PINCH beyond unit cost and tested PINCH against its non incremental counterpart GD. In my testing I have identified factors that indicate the performance of PINCH.\\

While I am slightly disappointed that PINCH did not perform as well as I had expected, I believe it is exactly the lackluster performance that motivated me to really dig into PINCH and figure out its strengths and weaknesses. Even though PINCH performs worse than GD on average, I have identified one group of domains for which PINCH is more likely to outperform GD. That group being domains with a low average number of preconditions per action and/or a high ratio of variables in relation to actions.\\

The biggest hindrance that PINCH unfortunately fails to overcome is the overhead associated with its priority queue. If I were to try to improve the algorithm my first instinct would be to adjust the priority queue such that it only stores variables like GD. However, I do not know if it is possible to make that adjustment and keep all or most of the good properties of PINCH.\\

It is promising that, despite its priority queue, PINCH still manages to beat GD in a good number of the covered instances. Even in its current form PINCH should prove useful if used for domains with the desired properties. While GD may have come out victorious at the end of my evaluation, I hope to have provided good enough reasons to argue for the benefits of PINCH. If the issue with its priority queue could be resolved, I believe that PINCH could be a fantastic option for the computation of $h^a^d^d$.