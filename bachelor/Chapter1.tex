% !TEX root = ../Thesis.tex
\chapter{Introduction}
{\small
Heuristic search planners like HSP 2.0 \cite{hsp2}, FF \cite{ff} or Fast Downward \cite{fast} transform planning problems described in a planning domain language (PDDL) into heuristic search problems. They derive heuristic values from the encoding of the planning problem and use these values to perform a heuristic forward or backward search in the space of world states. The goal is to find a path from a given initial state to a goal state. There are many different ways of extracting informed heuristic values from encoded planning problems. One of the most successful approaches is to consider a version of the planning problem where all delete effects of all actions are ignored (often referred to as a relaxed planning problem) and use the approximation of the relaxed planning cost as the heuristic value. One such way of approximating the relaxed planning cost is the additive heuristic. \\

When finding a plan for a problem, heuristic search planners spend the majority of the planning time calculating heuristic values, for HSP 2.0 about 80\% of the planning time \cite{bonet}. It therefore makes sense that, if we want to speed up the planning time, changing the way the heuristic values are computed should yield the most beneficial results. \\

For the additive heuristic, there are many different methods of computing the heuristic values. Perhaps the most intuitive one is to construct a relaxed planning graph and compute a plan for said graph. A different method uses a process called value iteration which iteratively updates variable values until none change in an iteration. An improvement on the value iteration approach is to order the value updates such that it leads to a quicker solution. An alternative way of speeding up the calculation uses information from previously determined heuristic values by performing incremental calculations.  \\

The goal of this thesis is to explore a specific implementation of the additive heuristic. The PINCH (Prioritized, INCremental, Heuristic calculation) method, as described by Yaxin Liu, Sven Koenig \& David Furcy \cite{main}, aims to combine both variable ordering and the reuse of information of previous heuristic values (incremental calculation). It is said to dramatically improve planning times compared to most alternative implementations of the additive heuristic\cite{main}. My aim is to provide an in depth look at PINCH and to implement and test the PINCH method and describe both its benefits and failings.}